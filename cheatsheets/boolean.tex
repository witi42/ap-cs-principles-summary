\documentclass{article}
\usepackage{xcolor}
\usepackage{soul}
\usepackage{array}
\usepackage{amsmath}
\usepackage{booktabs}
\usepackage[top=0.5in, bottom=0.5in]{geometry}

\usepackage{listings}

\definecolor{codegreen}{rgb}{0,0.6,0}
\definecolor{codegray}{rgb}{0.5,0.5,0.5}
\definecolor{codepurple}{rgb}{0.58,0,0.82}
\definecolor{backcolour}{rgb}{0.95,0.95,0.92}

\lstdefinestyle{mystyle}{
    mathescape=true,
    backgroundcolor=\color{backcolour},   
    % commentstyle=\color{codegreen},
    % keywordstyle=\color{magenta},
    % numberstyle=\tiny\color{codegray},
    % stringstyle=\color{codepurple},
    % basicstyle=\ttfamily\footnotesize,
    % breakatwhitespace=false,         
    % breaklines=true,                 
    % captionpos=b,                    
    % keepspaces=true,                 
    % numbers=left,                    
    % numbersep=5pt,                  
    % showspaces=false,                
    % showstringspaces=false,
    % showtabs=false,                  
    % tabsize=2
}
\lstset{style=mystyle}



% Define custom highlighting commands
\newcommand{\hlc}[2][yellow]{ {\sethlcolor{#1} \hl{#2}} }

\begin{document}

\title{Boolean Expressions \& Conditionals}
\author{}
\date{}
\maketitle

\textbf{Warning:} This document is a work in progress.


\section*{Boolean Expressions}

\subsection*{Categories}

\begin{itemize}
    \item \textbf{Numbers, Strings, Variables}
    \item \textbf{\hlc[red!20]{Logical Operators}}
    \item \textbf{\hlc[yellow!40]{Relational Operators}}
    \item \textbf{\hlc[blue!20]{Parentheses}}
\end{itemize}

\subsection*{Relational Operators}

Relational operators compare two values and evaluate to either true or false.

\begin{tabular}{l|l|l}
\toprule
\textbf{Code Symbol} & \textbf{AP Symbol} & \textbf{Meaning} \\
\midrule
\hlc[yellow!40]{\texttt{==}} & \hlc[yellow!40]{$=$} & Equal to \\
\hlc[yellow!40]{\texttt{!=}} & \hlc[yellow!40]{$\neq$} & Not equal to \\
\hlc[yellow!40]{\texttt{>}} & \hlc[yellow!40]{$>$} & Greater than \\
\hlc[yellow!40]{\texttt{<}} & \hlc[yellow!40]{$<$} & Less than \\
\hlc[yellow!40]{\texttt{>=}} & \hlc[yellow!40]{$\geq$} & Greater than or equal to \\
\hlc[yellow!40]{\texttt{<=}} & \hlc[yellow!40]{$\leq$} & Less than or equal to \\
\bottomrule
\end{tabular}

\subsection*{Logical Operators}

Logical operators combine two or more boolean expressions and evaluate to either true or false.

\begin{tabular}{l|l|l}
\toprule
\textbf{Code Symbol} & \textbf{AP Symbol} & \textbf{Meaning} \\
\midrule
\hlc[red!20]{\texttt{\&\&}} & \hlc[red!20]{AND} & Logical AND (true only if both are true) \\
\hlc[red!20]{\texttt{||}} & \hlc[red!20]{OR} & Logical OR (true if at least one is true) \\
\hlc[red!20]{\texttt{!}} & \hlc[red!20]{NOT} & Logical NOT (true if false, false if true) \\
\bottomrule
\end{tabular}



\subsection*{Examples}

\begin{itemize}
    \item \textit{Example 1:}
    
    \hlc[white]{x} \ \hlc[yellow!40]{$>$} \ \hlc[white]{10} \ \hlc[red!20]{AND} \ \hlc[white]{y} \ \hlc[yellow!40]{$<$} \ \hlc[white]{20}

    True for $x = 11, y = 19$

    False for $x = 11, y = 21$
    
    \item \textit{Example 2:}
    
    \hlc[red!20]{NOT} \ \hlc[blue!20]{(} \hlc[white]{a} \ \hlc[yellow!40]{==} \ \hlc[white]{b} \hlc[blue!20]{)}

    True for $a = 5, b = 2$

    False for $a = 11, b = 11$
    
\end{itemize}

\subsection*{Side by Side Comparison}

\begin{tabular}{l|l|l}
\toprule
\textbf{Operation} & \textbf{JavaScript} & \textbf{AP Pseudocode} \\
\midrule
Assignment & x = 5 & x $\leftarrow$ 5 \\
Equal to & \hlc[white]{a} \ \hlc[yellow!40]{==} \ \hlc[white]{b} & \hlc[white]{a} \ \hlc[yellow!40]{=} \ \hlc[white]{b} \\
% Not equal to & \hlc[white]{a} \ \hlc[yellow!40]{!=} \ \hlc[white]{b} & \hlc[white]{a} \ \hlc[yellow!40]{!=} \ \hlc[white]{b} \\
Logical OR & \hlc[white]{a} \ \hlc[red!20]{\texttt{||}} \ \hlc[white]{b} & \hlc[white]{a} \ \hlc[red!20]{OR} \ \hlc[white]{b} \\
Logical AND & \hlc[white]{a} \ \hlc[red!20]{\texttt{\&\&}} \ \hlc[white]{b} & \hlc[white]{a} \ \hlc[red!20]{AND} \ \hlc[white]{b} \\
Logical NOT & \hlc[red!20]{\texttt{!}}\hlc[white]{a} & \hlc[red!20]{NOT} \ \hlc[white]{a} \\
\bottomrule
\end{tabular}

% \section*{Tasks}

% \begin{tabular}{c|p{4.5in}|p{1in}}
% \toprule
% & \textbf{Expression} & \textbf{Value} \\
% \midrule
% a & \hlc[white]{} & \hlc[white]{} \\[5ex]
% b & \hlc[white]{} & \hlc[white]{} \\[5ex]
% c & \hlc[white]{} & \hlc[white]{} \\[5ex]
% d & \hlc[white]{} & \hlc[white]{} \\[5ex]
% e & \hlc[white]{} & \hlc[white]{} \\[5ex]
% f & \hlc[white]{} & \hlc[white]{} \\[3ex]
% \bottomrule
% \end{tabular}

% \section*{Conditionals}

\section*{Conditionals (AP CSP 3.6 \& 3.7)}


\subsection*{Basic IF Statement}

This is the basic structure of an if statement:
\begin{lstlisting}[mathescape=true]
    IF (condition)
    {
        statements
    }
\end{lstlisting}
If the condition is true, the statements inside the if block are executed.

Here's an example:
\begin{lstlisting}[mathescape=true]
    score $\leftarrow$ 100
    IF (score $\ge$ 100)
    {
        bonus $\leftarrow$ score * 10
    }
\end{lstlisting}
This example will set bonus to 1000 because the condition is true.


\subsection*{IF ELSE Statements}

This is the structure of an if else statement:
\begin{lstlisting}
    IF (condition)
    {
        statements
    }
    ELSE
    {
        statements2
    }
\end{lstlisting}

If the condition is true, the statements inside the if block are executed.
If the condition is false, the statements2 inside the else block are executed.

Here's an example:
\begin{lstlisting}
    score $\leftarrow$ 90
    IF (score $\ge$ 100)
    {
        bonus $\leftarrow$ score * 10
    }
    ELSE
    {
        bonus $\leftarrow$ 0
    }
\end{lstlisting}

This example will set bonus to 0 because the condition is false.

\subsection*{IF ELSE IF (ELSE) Statements}

This is the structure of an if else if statement:
\begin{lstlisting}
    IF (condition)
    {
        statements
    }
    ELSE IF (condition2)
    {
        statements2
    }
    ELSE
    {
        statements3
    }
\end{lstlisting}
If condition is true, statements inside the if block are executed.\\
If condition is false, condition2 is checked.\\
If condition2 is true, statements2 inside the else if block are executed.\\
If condition2 is false, statements3 inside the else block are executed (if present).\\
We can have multiple else if statements, but only one else statement (or no else statement).

Here's an example:
\begin{lstlisting}
    score $\leftarrow$ 85
    IF (score $\ge$ 90)
    {
        grade $\leftarrow$ "A"
    }
    ELSE IF (score $\ge$ 80)
    {
        grade $\leftarrow$ "B"
    }
\end{lstlisting}

This example will set grade to "B".


\subsection{Multiple IF Statements}

Else connects its block to the previous one to build one unit. Of that whole structure that starts with a single if statement, only one block will be executed.

However, multiple if statements that are not connected with an else statement are executed independently. 
Here's an example:
\begin{lstlisting}
    score $\leftarrow$ 100
    count $\leftarrow$ 0
    IF (score $\ge$ 80)
    {
        count $\leftarrow$ count + 1
    }
    IF (score $\ge$ 90)
    {
        count $\leftarrow$ count + 1
    }
    count $\leftarrow$ count + 1
    IF (score $\ge$ 100)
    {
        count $\leftarrow$ count + 1
    }
\end{lstlisting}

This example will set count to 4 because all three conditions are true and we add 1 to count without a condition between the last two if statements.

\subsection*{Nested Conditionals}

We can have if statements inside other if statements.

Here's an example:
\begin{lstlisting}
    homeworkComplete $\leftarrow$ true
    score $\leftarrow$ 95
    IF (homeworkComplete)
    {
        IF (score $\ge$ 90)
        {
            grade $\leftarrow$ "A"
        }
        ELSE
        {
            grade $\leftarrow$ "B"
        }
    }
\end{lstlisting}
We check if homeworkComplete is true. Only if it is true, we check if score is greater than or equal to 90. In this case, grade will be set to "A".

\end{document}

